% Style.
\documentclass[letterpaper,portuguese,12pt,pdftex]{exam}

\usepackage{setspace}
\usepackage{lineno}
\usepackage[left=2.5cm,top=3cm,right=2.5cm]{geometry}

% Portuguese.
\usepackage[brazil]{babel}
\usepackage[T1]{fontenc}
\usepackage[utf8x]{inputenc}
\usepackage{textcomp}

% Font.
\usepackage{lmodern}

% Figures.
\usepackage{epsf,epsfig}

% Bibtex and extras.
\usepackage{natbib}
\usepackage{url}
\usepackage[bookmarks=false,colorlinks=true,urlcolor={green},linkcolor={green},pdfstartview={XYZ null null 1.22}]{hyperref}

% Math.
\usepackage{amssymb,amsmath}
\usepackage{mathtools}
\everymath{\displaystyle}

% Exam.
\addpoints
\printanswers
\usepackage{color}
\definecolor{SolutionColor}{rgb}{0.8,0.9,1}
\shadedsolutions
\renewcommand{\solutiontitle}{\noindent\textbf{Solução:}\par\noindent}
\pagestyle{headandfoot}
\footer{}{Página \thepage\ de \numpages}{}
\boxedpoints
\pointsinrightmargin
\pointpoints{ponto}{pontos}
\hqword{Questão}
\hpword{Pontos}
\hsword{Nota}
% \qformat{\textbf{Question\thequestion}\quad(\thepoints)\hfill}

% User commands.
\newcommand{\pd}[2]{\frac{\partial #1}{\partial #2}}

% PDF metadata.
\pdfinfo{% hyperref overrides this
  /Title    (Prova 02 -- Climatologia e Meteorologia)
  /Author   (Filipe Fernandes)
  /Creator  (Filipe Fernandes)
  /Producer (Filipe Fernandes)
  /Subject  (prova)
  /Keywords (prova, oceanografia)
}

% Front page.
\title{Prova 02 -- Climatologia e Meteorologia}
\author{Prof. Filipe Fernandes}
\date{14-Abril-2014}

\begin{document}
\maketitle
\doublespacing

\hbox to \textwidth{Nome e número de matrícula:\enspace\hrulefill}
\vspace{0.5cm}

\begin{minipage}{.8\textwidth} % 35
Esse exame incluí \numquestions\ questões. O número total de pontos é \numpoints.
\vspace{1cm}
{\small
\begin{itemize}
  \item Coloque seu nome em todas as folhas e numere as mesma colocando o
        número total de folhas. (Ex.: 1/4, 2/4, 3/4 e 4/4.)
  \item Coloque suas respostas apenas na folha de respostas (pode usar
        as folhas da prova como rascunho).
  \item Essa prova segue o Acordo Ortográfico da Língua Portuguesa de 1990 (em
        vigor no início de 2009).  Erros de ortografia e gramática
        serão descontados da sua nota final.
  \item A prova deve ser feita individualmente e sem consulta.
  \item Use caneta (preta ou azul) para responder as questões – qualquer questão
        respondida à lápis não será considerada na hora da correção.
  \item Leia atentamente todas as questões: a interpretação faz parte da prova.
        Dúvidas serão esclarecidas apenas após o término da mesma.
  \item Desligue e guarde o celular.  Celulares à vista serão recolhidos!
\end{itemize}
}
\end{minipage}

\newpage

\begin{questions}

  \question[5]
    Sabemos que uma {\bf Frente Fria} (FF) é uma massa de ar frio avançando
    sobre uma massa de ar quente.  Enquanto uma {\bf Frente Quente} (FQ) é uma
    massa de ar quente avançando sobre uma massa de ar frio.

    Dito isso, explique as diferenças quanto a formação de nuvens e a posição
    das chuvas relativamente a posição da frente para ambos cenários de FF e FQ.

    \begin{solution}
      {\bf FF}: em geral produz nuvens do tipo Cúmulos Nimbus com chuvas na
      região frontal ou logo à sua frente.\\

      {\bf FQ}: produz quase todo o espectro de nuvens devido a FQ ``escalar'' a
      massa de ar frio, assim gerando nuvens de baixas a altas enquanto o ar
      quente ascende sobre o frio.  Em geral chove ao longo de todo o caminho
      ascendente do ar quente e não apenas na região frontal.
    \end{solution}

  \question[2\half]
    Explique o que está {\bf errado} na frase: ``O tornado chegou pelo oceano
    atingindo toda a costa da Lousiana elevando o nível do mar e causando
    chuvas torrenciais por 3 dias.''

    \begin{solution}
      Tornados são fenômenos de pequena escala (metros/minutos), provavelmente
      o fenômeno em questão é um furacão.
    \end{solution}


  \question[2\half]
    Você foi encarregado de planejar a aquisição remota de dados de chuva e
    relâmpagos para a {\bf cidade de São Paulo}.  A sua disposição está um
    moderno sensor de micro-ondas que será instalado no mais novo satélite
    meteorológico BRAZUCA-1.  Os engenheiros aeroespaciais do INPE planejam uma
    {\bf Órbita Polar} para esse satélite.  Você concorda?  Argumente contra ou
    a favor desse plano.

    \begin{solution}
      Para melhor observar a cidade de São Paulo seria recomendado um satélite
      de órbita geo-estacionária e não polar.  Garantindo assim altas resolução
      temporal e espacial na Cidade.
    \end{solution}


  \question[2\half]
    Seu novo gerente na ACME\circledR{} (o anterior foi demitido depois do
    fiasco do reservatório de água!) te encarregou de fazer um mapeamento de
    {\bf cobertura de nuvens} na região do Estádio Urbano Caldeira.
    (Provavelmente para melhor planejar as datas dos jogos em relação a
    probabilidade de chuvas.) Porém, ele contratou um serviço de imagens de
    satélite sem te consultar!  Para sua surpresa, seu gerente achou apropriado
    economizar e comprar apenas as imagens do canal de {\bf Vapor d'água}.

    Crie {\bf argumentos} para convencer seu gerente que será impossível
    realizar o mapeamento das chuvas com esse recurso e elabore um plano
    {\bf alternativo} que seja similarmente econômico.

    \begin{solution}
      Vapor d'água não traz informações sobre nuvens!  Mas sim informações
      da circulação atmosférica superior.  Como a finalidade é mapear chuvas o
      melhor produto seria um radar de micro-ondas.  (também aceitas: Canal
      Visível e/ou Canal do Infravermelho + Modelos de previsão.)
    \end{solution}


  \question[5]
    Sabemos que os {\bf Ciclones Tropicais} têm início com as perturbações das
    ondas de Leste, e crescem se ``alimentando'' da umidade e {\bf calor} dos
    oceanos tropicas.

    Explique: Porque não é comum a ocorrência de Ciclones Tropicais na costa do
    {\bf Brasil}?

    \begin{solution}
      Como dito, os Ciclones Tropicais precisão de calor e umidade para crescer.
      O Oceano Atlântico Sul é frio na Costa de África devido às ressurgências,
      costeiras comuns em contornos Leste (onde começam as perturbações de
      Leste.)
    \end{solution}


  \bonusquestion[5]
    {\bf Pergunta bônus}.

    A ocorrência de Ciclones Tropicais é rara no Atlântico Sul.  Mas temos um
    fenômeno similar chamado Ciclone Extra-Tropical que ocorre com relativa
    frequência preferencialmente no inverno Austral.  Explique de onde vêm a
    energia para crescimento desse tipo de fenômeno meteorológico.

    ({\bf Atenção}!!! A pergunta bônus será considerada somente se a
    resposta estiver 100\% correta!)

      \begin{solution}
        A energia para o crescimento dos CET vêm do gradiente latitudinal de
        temperatura.
      \end{solution}




\end{questions}
\end{document}
