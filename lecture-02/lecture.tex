% Title page.
\title[Aula 02]{Climatologia e Meteorologia}
\subtitle{Introdução ao estudo do clima}
\author[Filipe Fernandes]{Filipe P. A. Fernandes}
\institute[unimonte]{Centro Universitário Monte Serrat}
\date[Fevereiro 2014]{17 de Fevereiro 2014}

\logo{\includegraphics[scale=0.15]{../common/university_logo.png}}

\begin{document}

% The title page frame.
\begin{frame}[plain]
  \titlepage
\end{frame}

\section*{Outline}
\begin{frame}
\tableofcontents
\end{frame}

\section{Introdução ao estudo do clima}
\subsection{Clima}

\begin{frame}
\frametitle{Estudo do Clima}
  \begin{itemize}[<+-| alert@+>]
  \item Clima é a uma medida do comportamento ``Médio'' de um padrão de
        variação na temperatura, umidade, pressão atmosférica, vento,
        precipitação e etc;
  \item A medida desse comportamento {\bf Médio} se dá para uma certa região por
        um período longo de tempo.
  \item Clima se diferencia de tempo no sentido que tempo descreve apenas
        condições em um curto espaço de tempo dessas mesma varição em uma certa
        região.
  \end{itemize}
\pause
{\scriptsize Clima (do Grego antigo ``klima'') significa inclinação.  Uma média
típica para definir clima é de 30 anos*.}
\end{frame}

\begin{frame}
\frametitle{Estudo do Clima}
  \begin{block}{}
    O clima e de uma região e regido pelo ``Sistema Climático'' que possuí
    {\bf 5}     componentes: {\bf Atmosfera}, {\bf Hidrosfera}, {\bf Criosfera},
    {\bf Superfície Terrestre}, e {\bf Biosfera}.
  \end{block}
\end{frame}

\begin{frame}
\frametitle{Clima de uma localidade}
  \begin{itemize}[<+-| alert@+>]
    \item Altitude;
    \item Latitude;
    \item Terreno;
    \item Proximidade de Corpos d'água (e suas correntes).
  \end{itemize}
\end{frame}

\begin{frame}
\frametitle{Clima de uma certa localidade}
  \begin{block}{}
    Climas podem ser classificados de acordo com uma média e ``alcance'' típico
    das diferentes variáveis, em geral temperatura e chuva.
  \end{block}
\pause
  \begin{block}{}
    O sistema de classificação mais utilizado foi desenvolvido pelo
    meteorologista russo Wladimir Köppen.
  \end{block}
\pause
{\scriptsize
\url{http://global.britannica.com/EBchecked/topic/322065/Wladimir-Koppen}

\href{http://en.wikipedia.org/wiki/K\%C3\%B6ppen_climate_classification}{http://en.wikipedia.org/wiki/Köppen\_climate\_classification}
}
\end{frame}

\begin{frame}
\frametitle{Clima de uma certa localidade}
  \begin{center}
    \shadowbox{\includegraphics[scale=0.6]{./figures/breathing-earth-0.png}}
  \end{center}
\end{frame}

\subsection{Paleoclima}
\begin{frame}
\frametitle{Paleoclima}
  \begin{block}{}
    É o estudo de climas passados.  Como não há observações diretas do clima
    antes do século 19, paleoclimas são inferidos com o uso de variáveis
    {\it proxies}.
  \end{block}
  \pause
  \begin{block}{}
    {\it Proxies} incluem evidência não-bióticas, como sedimentos
    encontrados no fundo de lagos, oceanos e testemunhos de gelo; e evidências
    bióticas, como anéis de árvores e corais.
  \end{block}
  \pause
  \begin{block}{}
    Hoje em dia também se faz uso de modelos matemáticos de clima passado, presente e futuro.
  \end{block}
\end{frame}

\begin{frame}
\frametitle{Corrente do Golfo, montanhas e o Inverno Europeu.}
\end{frame}

\section{Conceitos e definições: clima e tempo}
\subsection{Clima é o que você espera, tempo é o que você recebe!}

\begin{frame}
\frametitle{Clima e tempo}
  \begin{block}{}
    Clima é o que você espera, tempo é o que você recebe!
  \end{block}
\end{frame}

\begin{frame}
\frametitle{Tempo}
  \begin{itemize}[<+-| alert@+>]
    \item É o estado da atmosfera;
    \item Quente ou Frio;
    \item Úmido ou seco;
    \item Calmo ou tempestuoso;
    \item Claro ou nebuloso.
  \end{itemize}
\end{frame}

\begin{frame}
\frametitle{Tempo}
  \begin{block}{}
    Tempo é gerado pela diferença da pressão do ar (temperatura e umidade) de
    um lugar para outro.
  \end{block}
\pause
  \begin{block}{}
    Tempo tem escala temporal de dia-a-dia (as vezes hora-a-hora).
  \end{block}
\pause
  \begin{block}{}
     A atmosfera é um sistema de natureza caótica.
  \end{block}
\hfill \url{http://en.wikipedia.org/wiki/Butterfly_effect}
\end{frame}

\begin{frame}
\frametitle{Fenômenos do tempo}
Ocorrem na troposfera.
  \begin{itemize}[<+-| alert@+>]
    \item Vento;
    \item Nuvem;
    \item Chuva;
    \item Neve;
    \item Neblina;
    \item Tempestade de poeira/areia.
  \end{itemize}
\pause
{\scriptsize Menos conhecidos incluem: tornados, furacões, tufões,
tempestade de gelo.}
\end{frame}

\begin{frame}
\frametitle{Sol, ângulos, instabilidades e os Ciclones Extratropicais.}
\end{frame}

\begin{frame}
\frametitle{Pergunta}
  \begin{block}{}
    O ``Great Red Spot'' de Júpiter é um fenômeno de tempo ou fenômeno
    climático?
  \end{block}
\end{frame}

\begin{frame}
  \begin{center}
    \movie[showcontrols=true]{\shadowbox{\phantom{\rule{8cm}{5cm}}}}
  {./figures/Cassino_2014_09_02-vzcfUp4OZGQ.mp4}
  \end{center}
\end{frame}

\begin{frame}
\frametitle{Tempo e intemperismo.}
  \begin{itemize}[<+-| alert@+>]
    \item Processo responsável por ``moldar'' a Terra;
    \item Quebra rochas e solo em fragmentos menores;
    \item Chuva Ácida (como fenômeno natural);
    \item Erosão e o Transporte de Sal.
  \end{itemize}
\end{frame}

\begin{frame}
\frametitle{Principais Sistemas de Vento e Pressão (e sua relação com o tempo).}
{\tiny
\begin{table}
    \begin{tabular}{lllll}
    Região                           & Nome                     & Pressão & Ventos                     & Tempo                     \\
        \hline
    Equador (0\textdegree{})         & Marasmo (ZCIT)           & Baixa   & Fracos, sem direção        & Chuvas abundantes         \\
    0\textdegree{}--30\textdegree{}  & Alísios                  & -       & Nordeste (N), Sudeste (S)  & Verão úmido, Inverno seco \\
    30\textdegree{}                  & ``Latitudes de Cavalo''  & Alta    & Fracos, sem direção        & Seco                      \\
    30\textdegree{}--60\textdegree{} & Ventos Oeste             & -       & Sudoeste (N), Noroeste (S) & Verão seco, Inverno úmido \\
    60\textdegree{}                  & Frente Polar             & Baixa   & Variáveis                  & Tempestuoso               \\
    60\textdegree{}--90\textdegree{} & Polares de Leste         & -       & Nordeste (N), Sudeste (S)  & Ar frio polar             \\
    90\textdegree{}                  & Polos                    & Alta    & Sul (N), Norte (S)         & Ar frio e seco            \\
    \hline
    \end{tabular}
\end{table}
}
\end{frame}


\begin{frame}
\frametitle{Células de Vento}
  \begin{center}
    \shadowbox{\includegraphics[scale=0.4]{./figures/Earth_Global_Circulation.jpg}}
  \end{center}
\end{frame}


\subsection{O tempo, o clima e o homem}
\begin{frame}
\frametitle{O tempo, o clima e o homem}
  \begin{itemize}[<+-| alert@+>]
    \item O Clima rege a distribuição dos homem no planeta;
    \item Porque as zonas costeiras são mais densamente povoadas?
    \item Porque as zonas desérticas são esparsamente povoadas?
\end{itemize}
\end{frame}


\begin{frame}
\frametitle{O tempo, o clima e o homem}
  \begin{itemize}[<+-| alert@+>]
    \item O Tempo também pode afetar o homem (de uma forma quase tão dramática
          quanto o clima.)
    \item Os Franceses declararam a Flórida como perdida em 1565 quando um
          furacão destruiu a frota Francesa, permitindo os espanhóis
          conquistá-la.
    \item Recentemente o Furacão Katrina redistribuiu mais de um milhões de
          pessoas da costa do Golfo do México para o interior.
\end{itemize}
\end{frame}

\section{Conceitos e definições: meteorologia e climatologia}
\subsection{Meteorologia}
\begin{frame}
\frametitle{Meteorologia}
  \begin{block}{}
    É a ciência multidisciplinar que estuda a Atmosfera.
  \end{block}
\pause
  \begin{block}{}
    Meteorologia vem do grego ``metéoros'' (ou meta) que significa alto (no céu,
    acima) e ``logia'' que significa estudo.
  \end{block}
\end{frame}

\begin{frame}
\frametitle{Escalas}
  \begin{itemize}[<+-| alert@+>]
    \item Micro-escala: é o estudo de fenômenos atmosféricos
          $< 1$ km. (Tempestades individuais, nuvens etc.)
    \item Meso-escala: é o estudo de fenômenos atmosféricos $> 1$ km mas menores
          que a escala sinótica.  As escalas temporais vão de menos de um dia até
          semanas. (Frentes, bandas de precipitação na zona tropical, ciclones
          extratropicais, brisa marinha etc.)
  \end{itemize}
\end{frame}

\begin{frame}
\frametitle{Escalas}
  \begin{itemize}[<+-| alert@+>]
    \item Escala Sinótica: Geralmente é a definição de uma grande área dinâmica
          referenciada em coordenadas horizontais com respeito ao tempo.
          (Ciclones extratropicais, cavas e cristas baroclínicas, zonas frontais,
          e, até certo ponto, os ``jet streams''.
    \item Escala Global: é o estudo de padrões de tempo relacionados ao
          Transporte de calor dos trópicos até os polos.  Assim como oscilações
          de larga escala (na ordem de meses), como as oscilações ``Madden-
          Julian'', ou anos, como o El Niño-Oscilação Sul e oscilação decadal do
          Pacífico.
  \end{itemize}
\end{frame}

\begin{frame}
\frametitle{Cartas Sinóticas -- Hoje}
  \begin{center}
    \shadowbox{\includegraphics[scale=0.35]{./figures/C14021712.jpg}}
  \end{center}
{\scriptsize \url{https://www.mar.mil.br/dhn/chm/meteo/prev/cartas/C14021712.jpg}}
\end{frame}


\begin{frame}
\frametitle{Cartas Sinóticas -- Semana passada}
  \begin{center}
    \shadowbox{\includegraphics[scale=0.35]{./figures/C14021012.jpg}}
  \end{center}
{\scriptsize \url{https://www.mar.mil.br/dhn/chm/meteo/prev/cartas/C14021012.jpg}}
\end{frame}


\subsection{Climatologia}

\begin{frame}
\frametitle{Climatologia}
  \begin{block}{}
    Climatologia é o estudo do clima, cientificamente definido como condições
    do Tempo promediadas por um certo período de tempo.
  \end{block}
\pause
  \begin{block}{}
    Climatologia nasceu do estudo de climas passados e suas mudanças.  Hoje
    temos diversos índices de clima que surgiram do estudo do clima: ENSO, NAO,
    PDO, MJO, IPO etc.
  \end{block}
{\scriptsize \url{http://en.wikipedia.org/wiki/Climatology\#Indices}}
\end{frame}


\section{Datas das provas}
\begin{frame}
\frametitle{Novas datas provas}
{\scriptsize
  \begin{longtable}[c]{@{}cllr@{}}
  Aula & Data & Conteúdo & Lista/Prova \\
  \midrule\endhead
  01 & 2014-02-03 & {\bf NA}    & \\
  02 & 2014-02-10 & Tópico 1    & \\
  03 & 2014-02-17 & Tópico 2    & \\
  04 & 2014-02-24 & Tópico 3    & \\
  05 & 2014-03-10 & Tópico 4    & \\
  06 & 2014-03-17 & Tópico 5    & \\
  \color{red}{07} & \color{red}{2014-03-24} & \color{red}{Tópico 1-5}  & \color{red}{P1\_1} \\
  08 & 2014-03-31 & Tópico 6    & \\
  09 & 2014-04-07 & Tópico 7    & \\
  10 & 2014-04-14 & Tópico 6-7  & P1\_2 \\
  11 & 2014-04-28 & Tópico 9    & \\
  12 & 2014-05-05 & Tópico 10   & \\
  \color{red}{13} & \color{red}{2014-05-12} & \color{red}{Tópico 9-10}   & \color{red}{P2\_1}\\
  14 & 2014-05-19 & Tópico 11   & \\
  15 & 2014-05-26 & Tópico 12   & \\
  16 & 2014-06-02 & Tópico 1-13 & T1 (Seminários) \\
  17 & 2014-06-09 & Tópico 8-12 & P2\_2 \\
  18 & 2014-06-16 & Tópico 1-12 & Revisão PA \\
  19 & 2014-06-30 & Tópico 1-12 & PA \\
  \bottomrule
  \end{longtable}
  }
\end{frame}

\begin{frame}
  \frametitle{Dever de casa -- 02}
  Aula prática da próxima semana.
\end{frame}

\end{document}
