% Title page.
\title[Aula 08]{Climatologia e Meteorologia}
\subtitle{Interpretação de imagens de satélites}
\author[Filipe Fernandes]{Filipe P. A. Fernandes}
\institute[unimonte]{Centro Universitário Monte Serrat}
\date[Abril 2014]{7 de Abril 2014}

\logo{\includegraphics[scale=0.15]{../common/university_logo.png}}

\begin{document}

% The title page frame.
\begin{frame}[plain]
  \titlepage
\end{frame}

\section*{Outline}
\begin{frame}
\tableofcontents
\end{frame}

\section{Interpretação de imagens de satélites}

\subsection{Sensoriamento Remoto}

\begin{frame}
\frametitle{Sensoriamento Remoto}
  \begin{block}{Lillesand \& Kiefer (1987)}
    ``... a ciência e arte de receber informações sobre um objeto, uma área ou
    fenômeno pela análise dos dados obtidos de uma maneira tal que não
    haja contato direto com este objeto, esta área ou este fenômeno''
  \end{block}

\end{frame}

\begin{frame}
\frametitle{Satélites}
  \begin{itemize}[<+-| alert@+>]
    \item Órbita Polar;
    \item Órbita Geo-estacionária
    \item Órbita Equatorial
    \item Observação passiva
    \item Observação ativa
    \item Resolução: Espacial, Temporal, Espectral, Radiométrica.
  \end{itemize}
\end{frame}

\begin{frame}
\frametitle{Satélites}
  \begin{itemize}[<+-| alert@+>]
    \item Medições em comprimentos de onda específicos: uso de filtros;
    \item Intervalo espectral: canais ou bandas;
    \item Grande parte dos satélites meteorológicos: visível, infravermelho e
          canal do vapor d'água
    \item Alguns têm canais adicionais desde o ultravioleta (100 a 400 nm) às
          microondas (0,15 a 6,0 cm ou 200 a 5 GHz)
  \end{itemize}
\end{frame}


\begin{frame}
\frametitle{Comprimentos de Ondas}
  \begin{center}
    \includegraphics[scale=0.5]{./figures/wavelength.png}
  \end{center}
\end{frame}


\begin{frame}
\frametitle{Satélites}
  \begin{center}
    \includegraphics[scale=0.5]{./figures/satelites.png}
  \end{center}
\end{frame}

\begin{frame}
\frametitle{Observações}
  \begin{center}
    \includegraphics[scale=0.5]{./figures/oberservacoes.png}
  \end{center}
\end{frame}


\subsection{Satélites Meteorológicos}
\begin{frame}
\frametitle{Imagens}
  \begin{block}{}
  Cobertura de nuvens:\\
  Desde a primeira imagem de satélite até os dias de hoje é a característica
  mais marcante na imagem.
  \end{block}
  \pause
  \begin{block}{}
    Meteorologista previsor:\\
    Uso principal é previsão de tempo, a análise em geral se limita ao estudo
    da cobertura de nuvens.
  \end{block}
\end{frame}


\subsection{Análise de Imagens}
\begin{frame}
\frametitle{Nuvens}
  \begin{itemize}[<+-| alert@+>]
    \item Nuvens podem parecer aleatórias tanto na forma quanto na distribuição.
    \item Entretanto, elas se formam como o resultado de interações específicas
          entre diferentes fatores meteorológicos.
    \item Nuvens que se formam sob condições similares podem ser classificadas
          em categorias individuais com padrões distintos nas imagens de
          satélites.
    \item Ao conhecer tais padrões, é possível identificar os tipos de nuvens
          presentes em uma imagem de satélite.
  \end{itemize}
\end{frame}


\begin{frame}
\frametitle{Nuvens}
  \begin{itemize}[<+-| alert@+>]
    \item Refletância de nuvens: a partir da qual se pode inferir sua
          profundidade e composição (se água ou gelo)
    \item Textura: Lisa ou fibrosa.
    \item Forma dos elementos: Regular ou não.
    \item Padrão dos elementos: Associado com a topografia, fluxo de ar,
          cisalhamento vertical e horizontal do vento.
    \item Tamanho.
    \item Estrutura vertical: A presença de sombras abaixo das nuvens por
          exemplo.
  \end{itemize}
\end{frame}


\begin{frame}
\frametitle{Análise de imagens}
  \begin{itemize}[<+-| alert@+>]
    \item Deve ser acompanhada de outras informações:
    \begin{enumerate}
      \item Data e hora da imagem
      \item Área geográfica coberta
      \item Escala de cinza utilizada
      \item Outros dados observacionais e cartas sinóticas
    \end{enumerate}
    \item Continuidade no tempo: análise e comparação de sequência de imagens
          -- detectar mudanças e/ou confirmar a interpretação
  \end{itemize}
\end{frame}


\begin{frame}
\frametitle{Análise de imagens}
  \begin{itemize}[<+-| alert@+>]
    \item Comparação de imagens de diferentes canais:
    \item Em geral, a informação não é redundante, mas complementar.
    \item Ajuda a remover incertezas ou ambiguidades na interpretação dos
          processos e a distinguir características de nuvens, fumaça, e da
          superfície mais facilmente.
  \end{itemize}
\end{frame}


\begin{frame}
\frametitle{Entretanto...}
  \begin{block}{}
    Observações em superfície ainda são fundamentais para a boa análise da evolução do tempo.
  \end{block}
  \pause
  \begin{block}{}
    Os produtos de satélites representam ferramenta adicional para interpretar
    e acompanhar as mudanças observadas na atmosfera.
  \end{block}
\end{frame}


\begin{frame}
\frametitle{Visível}
  \begin{itemize}[<+-| alert@+>]
    \item Em geral, é a imagem com maior resolução espacial.
    \item Radiação solar refletida pela superfície e atmosfera (só durante o
          dia).
    \item Áreas com tons mais claros indicam superfícies com alta refletância.
    \item Nuvens espessas são brancas, nuvens intermediárias apresentam
          coloração cinza.
    \item Oceano é quase negro.
  \end{itemize}
\end{frame}


\begin{frame}
\frametitle{Visível -- 2014--03--26--17:45 GOES-13}
  \begin{center}
    \includegraphics[scale=0.2]{./figures/GOES/vis_1403261745G13I01.png}
  \end{center}
\end{frame}


\begin{frame}
\frametitle{Visível -- Nuvens}
  \begin{itemize}[<+-| alert@+>]
    \item Nuvens que aparecem mais brilhantes têm maior albedo: são mais altas,
          possuem alto conteúdo de água ou gelo e têm gotículas menores.
    \item Nuvens com coloração cinza são as rasas, com menor conteúdo de água
          ou gelo e com gotas maiores.
    \item Portanto, as Cb são facilmente observáveis ao passo que as cirrus não
          podem ser vistas claramente.
  \end{itemize}
\end{frame}


\begin{frame}
\frametitle{Infravermelho}
  \begin{block}{}
    Canais mais frequentes: entre 1 e 30 $\mu$m.
  \end{block}
  \pause
  \begin{block}{}
    Nos satélites
    meteorológicos: 10 a 12,5 $\mu$m -- janela atmosférica.
  \end{block}
\end{frame}


\begin{frame}
\frametitle{Infravermelho}
  \begin{itemize}[<+-| alert@+>]
    \item Radiação emitida pela superfície e atmosfera.
    \item Proporcional à temperatura.
    \item Superfície ou topo de nuvens.
    \item Regiões com temperaturas menores aparecem mais brancas.
    \item Superfícies mais quentes aparecem com tonalidades mais escuras.
  \end{itemize}
\end{frame}


\begin{frame}
\frametitle{Infravermelho}
  \begin{itemize}[<+-| alert@+>]
    \item Medições também à noite.
    \item Cobertura contínua da evolução das nuvens num período de 24 horas.
    \item No processamento de imagens:
      \begin{enumerate}
        \item Quanto maior a radiância medida, mais brilhante o pixel.
        \item Padrão é invertido: quanto maior a radiância medida, mais escuro é
              o pixel.
        \item Nuvens são mais brancas e superfícies são escuras.
      \end{enumerate}
  \end{itemize}
\end{frame}



\begin{frame}
\frametitle{Infravermelho}
  \begin{itemize}[<+-| alert@+>]
    \item Canal 4 (11 $\mu$m)
    \item Quanto mais alto a nuvem mais frio será o topo;
    \item Nuvens baixas que são mais quentes aparecem mais escuras nas imagens
          do infravermelho;
    \item Nuvens altas são mais frias e portanto mais brilhantes em imagens do
          infravermelho.
  \end{itemize}
\end{frame}


\begin{frame}
\frametitle{IV -- 2014--03--26--17:45 GOES-13}
  \begin{center}
    \includegraphics[scale=0.2]{./figures/GOES/ir4_1403262345G13I04.png}
  \end{center}
\end{frame}


\begin{frame}
\frametitle{Infravermelho}
  \begin{block}{}
    Não é possível distinguir a textura das nuvens.
  \end{block}
  \pause
  \begin{block}{}
    Não é possível discriminar nuvens baixas de superfície oceânica ou nevoeiro
    de superfície continental, devido ao baixo contraste térmico entre eles.
  \end{block}
\end{frame}


\begin{frame}
\frametitle{Vapor d'água}
  \begin{itemize}[<+-| alert@+>]
    \item Também disponível em qualquer hora do dia.
    \item Banda de absorção do vapor d'água, centralizada em 6--7 $\mu$m.
    \item Como o conteúdo de vapor d'água decresce com a altura, a maior
          contribuição para a radiância medida pelo satélite é proveniente
          dos níveis médios e altos da troposfera.
    \item Regiões mais brilhantes apresentam alta umidade na alta troposfera
    \item Regiões escuras são aquelas nas quais a alta
          troposfera está muito seca.
  \end{itemize}
\end{frame}



\begin{frame}
\frametitle{Vapor d'água}
  \begin{center}
    \includegraphics[scale=0.2]{./figures/GOES/ir3_1403262045G13I03.png}
  \end{center}
\end{frame}


\begin{frame}
\frametitle{Vapor d'água}
  \begin{center}
    \includegraphics[scale=0.2]{./figures/GOES/ir3_1403272345G13I03.png}
  \end{center}
\end{frame}


\begin{frame}
\frametitle{Infravermelho vs Vapor d'água vs Visível}
  \begin{block}{}
    Imagens no visível e infravermelho auxiliam na análise e previsão do tempo a
    partir das características das nuvens presentes nas imagens.
  \end{block}
  \pause
  \begin{block}{}
    Imagens no canal do vapor d'água podem ser utilizadas para observar padrões
    de circulação de larga escala mesmo na ausência de nuvens!
  \end{block}
\end{frame}


\begin{frame}
\frametitle{Microondas}
  \begin{itemize}[<+-| alert@+>]
    \item Radiação nessa região espectral é sensível a:
      \begin{enumerate}
        \item Precipitação.
        \item Água ou gelo de nuvem.
        \item Vapor d'água.
        \item Umidade do solo.
        \item Temperatura da superfície e da atmosfera.
        \item Velocidade do vento à superfície do oceano.
      \end{enumerate}
  \end{itemize}
\end{frame}


\begin{frame}
\frametitle{Microondas}
  \begin{center}
    \includegraphics[scale=0.5]{./figures/Ike_microwave_1030z7Sep.png}
  \end{center}
\end{frame}


\begin{frame}
  \frametitle{Leitura Extra}
  \begin{block}{}
    \url{http://pt.allmetsat.com/interpretacao.php}
  \end{block}
\end{frame}



\subsection{Exemplos}
\begin{frame}
\frametitle{Exemplos}
  \begin{center}
    \includegraphics[scale=0.5]{./figures/vis_ir4_nuvens_1.png}
  \end{center}
\end{frame}

\begin{frame}
\frametitle{Exemplos}
  \begin{center}
    \includegraphics[scale=0.5]{./figures/vis_ir4_nuvens_2.png}
  \end{center}
\end{frame}

\begin{frame}
\frametitle{Exemplos}
  \begin{center}
    \includegraphics[scale=0.5]{./figures/vis_ir4_nuvens_3.png}
  \end{center}
\end{frame}

\begin{frame}
\frametitle{Exemplos}
  \begin{center}
    \includegraphics[scale=0.5]{./figures/vis_ir4_nuvens_4.png}
  \end{center}
\end{frame}


\end{document}
