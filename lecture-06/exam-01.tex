% Style.
\documentclass[letterpaper,portuguese,12pt,pdftex]{exam}

\usepackage{setspace}
\usepackage{lineno}
\usepackage[left=2.5cm,top=3cm,right=2.5cm]{geometry}

% Portuguese.
\usepackage[brazil]{babel}
\usepackage[T1]{fontenc}
\usepackage[utf8x]{inputenc}
\usepackage{textcomp}

% Font.
\usepackage{lmodern}

% Figures.
\usepackage{epsf,epsfig}

% Bibtex and extras.
\usepackage{natbib}
\usepackage{url}
\usepackage[bookmarks=false,colorlinks=true,urlcolor={green},linkcolor={green},pdfstartview={XYZ null null 1.22}]{hyperref}

% Math.
\usepackage{amssymb,amsmath}
\usepackage{mathtools}
\everymath{\displaystyle}

% Exam.
\addpoints
\printanswers
\usepackage{color}
\definecolor{SolutionColor}{rgb}{0.8,0.9,1}
\shadedsolutions
\renewcommand{\solutiontitle}{\noindent\textbf{Solução:}\par\noindent}
\pagestyle{headandfoot}
\footer{}{Página \thepage\ de \numpages}{}
\boxedpoints
\pointsinrightmargin
\pointpoints{ponto}{pontos}
\hqword{Questão}
\hpword{Pontos}
\hsword{Nota}
% \qformat{\textbf{Question\thequestion}\quad(\thepoints)\hfill}

% User commands.
\newcommand{\pd}[2]{\frac{\partial #1}{\partial #2}}

% PDF metadata.
\pdfinfo{% hyperref overrides this
  /Title    (Prova 01 -- Climatologia e Meteorologia)
  /Author   (Filipe Fernandes)
  /Creator  (Filipe Fernandes)
  /Producer (Filipe Fernandes)
  /Subject  (prova)
  /Keywords (prova, oceanografia)
}

% Front page.
\title{Prova 01 -- Climatologia e Meteorologia}
\author{Prof. Filipe Fernandes}
\date{24-Mar-2014}

\begin{document}
\maketitle
\doublespacing

\vspace{1cm}
\hbox to \textwidth{Nome e número de matrícula:\enspace\hrulefill}
\vspace{1cm}

\begin{minipage}{.8\textwidth}
Esse documento incluí \numquestions\ questões. O número total de pontos é \numpoints.
\vspace{1cm}

A prova segue o Acordo Ortográfico da Língua Portuguesa de 1990 (em vigor no
início de 2009).  Erros de ortografia e gramática serão descontados da sua nota
final.

\vspace{1cm}

A prova deve ser feita individualmente e sem consulta.  O aluno deverá usar
{\bf caneta} (preta ou azul) para responder as questões -- qualquer questão
respondida à lápis não será considerada na hora da correção!  Coloque seu nome
em {\bf todas} as folhas e numere as mesmas informando número total de folhas
utilizadas (Ex.: 1/4, 2/4, 3/4, 4/4.)

\vspace{1cm}

Leia atentamente todas as questões: a interpretação faz parte da prova e dúvidas
serão esclarecidas apenas após o término da mesma.

\end{minipage}

\newpage

\begin{questions}

\question[2]
É comum ouvir a seguinte afirmativa ``O clima esta chuvoso e o tempo
subtropical.''  Para evitar erros como esse explique as diferenças e/ou
semelhanças entre os pares de conceitos abaixo.  (Dê exemplos quando for
adequado.)

\begin{itemize}
  \item[a)] Tempo vs Clima.
  \item[c)] Tempo vs Intemperismo.
  \item[b)] Paleoclima vs Clima vs Climatologia.
\end{itemize}

\begin{solution}
\begin{itemize}
  \item[a)] Tempo mudanças rápidas (condições meteorológicas no momento)
            Clima é a ``média'' das condições meteorológicas ao longo de
            $~$30 anos de uma certa região.
  \item[b)] Intemperismo é o efeito do tempo na Terra (rochas, sedimentos etc).
  \item[c)] Climatologia é o estudo do clima, Clima é média da condição
            meteorológica de uma região e Paleoclima é o estado do clima no
            passado da Terra (que pode ser muito diferente do presente).
\end{itemize}
\end{solution}

\question[2\half]
Descreva como se estabelece o padrão médio da variação meridional
(latitudinal) da temperatura no globo.  Escolha pelo menos duas anomalias que
contradizem a sua resposta acima e explique. (Dica: Use a figura abaixo como
guia.)

% \begin{center}
%    \includegraphics[scale=0.55]{../lecture-04/figures/latitude.png}
% \end{center}

\begin{solution}
O padrão médio latitudinal de temperatura se dá devido a curvatura da Terra,
onde, assumindo que os raios solares chegam sempre paralelos à Terra, as áreas
próximas aos polos recebem uma ``densidade'' menor de raios que a parte
Equatorial.

Anomalias: Figura mostra temperaturas mais baixas que seus arredores na zona
Sub-Tropical (como na Costa da África) devido à ressurgência costeira e em zonas
de alta altitude (como nos Andes).
\end{solution}


\question[3]
Esquematize e explique como seria a circulação dos ventos na Troposfera
caso o planeta Terra {\bf não} girasse no próximo eixo.  Faça um paralelo com a
circulação que realmente observamos.

\begin{solution}
  Teríamos duas grandes células de circulação do Equador aos Polos onde: ar
  quente ascenderia na região equatorial, circulando o hemisfério até o Polo
  onde se resfriaria, desceria, iniciando a circulação de retorno ao Equador.

  Tal padrão, seguindo apenas a convecção devido à distribuição de calor, não é
  observado devido à rotação da Terra.  A Força de Coriolis, deflete o ar para a
  esquerda no HS e direita HN fazendo que que essa ``grande célula'' se quebre
  em células pequenas.
\end{solution}

\question[2\half]
Como determinamos as origens das massas de ar?  (Comente sobre as variáveis
necessárias para descrevê-las.)

\begin{solution}
  As características das massas de ar são definidas pela da superfície que está
  abaixo dela em sua origem.  Essas características são definidas definidas pela
  sua temperatura e umidade (vapor d'água).
\end{solution}

\question[1\half]
A Figura abaixo representa a distribuição das massas de ar no Brasil durante
o Verão.  Sabendo que as massas de origem oceânica e a Equatorial Continental
são quentes e úmidas, comente como seria o clima esperado nas regiões Norte e
Nordeste do Brasil.

\begin{solution}
  Ambas massas quentes e úmidas são direcionadas para o litoral Norte e Nordeste
  Brasileiro.  Porém, apenas o litoral tem clima quente e úmido, o interior do
  Nordeste tem climas de árido a semi-árido.  Esse clima seco é devindo as
  chuvas ocorrem antes de chegar nessas áreas em função de convecção ou
  orografia local.  Fazendo com que a massa de ar chegue seca e quente no
  semi-árido brasileiro.
\end{solution}

% \begin{center}
%   \includegraphics[scale=1]{../lecture-04/figures/massa_de_ar-verao.png}
% \end{center}

\question[2]
A ``friagem'' consiste na queda brusca de temperatura na Amazônia Ocidental.
Sobre esse fenômeno pode-se afirmar que:

\begin{itemize}
  \item[a)] O relevo baixo de planície facilita a incursão de massa e de ar
            frio que atinge a Amazônia.
  \item[b)] A massa de ar responsável pela ocorrência de friagem é a tropical
            atlântica.
  \item[c)] A friagem ocorre no inverno.
\end{itemize}

De acordo com as afirmativas, assinale a alternativa correta.

\begin{itemize}
  \item[a)] Somente o item (a) está correto.
  \item[b)] Os itens (b) e (c) estão corretos.
  \item[c)] Os itens (a) e (b) estão corretos
  \item[d)] Os itens (a) e (c) estão corretos
  \item[e)] Todos os itens estão corretos
\end{itemize}

\begin{solution}
  Item (d).
\end{solution}


\question[2]
Imagine uma destruição hipotética da floresta amazônica, onde todas as
árvores foram cortadas para dar lugar a um estacionamento de um Mega Shopping.
Qual destas consequências climáticas parece ser a mais provável?

\begin{itemize}
  \item[a)] Aumento dos índices pluviométricos e nas médias térmicas.
  \item[b)] Diminuição dos ventos, especialmente dos alísios do hemisfério Sul.
  \item[c)] Diminuição das chuvas, já que a maior parte da umidade atmosférica
            da região se deve à evapotranspiração das plantas.
  \item[d)] Diminuição das nuvens e aumento na velocidade dos ventos pela
            ausência de obstáculos no seu caminho.
  \item[e)] Aumento das precipitações e diminuição da temperatura.
\end{itemize}

\begin{solution}
  Item (c).
\end{solution}

\question[2]
Você foi contratado pela empresa ACME\circledR{} para avaliar terrenos para uma
futura construção de um reservatório de água do tipo represa.  (Uma tentativa de
acabar com o racionamento de água que assola a região de São Paulo no momento.)

Cite pelo menos 3 fatores {\bf Climáticos--regionais} que te levariam a
escolher um terreno apropriado.

\begin{solution}
Resposta livre dentro dos conceitos de captação de água, chuvas convectivas e
orográficas (ou de relevo).
%   A escolha do terreno deve ser em um local:
%   \begin{itemize}
%     \item que receba águas de chuva, ou seja, antes de obstáculos orográficos;
%     \item de fácil escoamento para a região urbana;
%     \item que capte água de rios e afluentes da região.
%   \end{itemize}

\end{solution}

\end{questions}

\end{document}
