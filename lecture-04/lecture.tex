% Title page.
\title[Aula 04]{Climatologia e Meteorologia}
\subtitle{Introdução ao estudo do clima}
\author[Filipe Fernandes]{Filipe P. A. Fernandes}
\institute[unimonte]{Centro Universitário Monte Serrat}
\date[Março 2014]{10 de Março 2014}

\logo{\includegraphics[scale=0.15]{../common/university_logo.png}}

\begin{document}

% The title page frame.
\begin{frame}[plain]
  \titlepage
\end{frame}

\section*{Outline}
\begin{frame}
\tableofcontents
\end{frame}

\section{As massas de ar atuantes no Brasil}

\subsection{Revisão}
\begin{frame}
\frametitle{As massas de ar atuantes no Brasil.}
  \begin{block}{Revisão}
    \begin{itemize}[<+-| alert@+>]
      \item Latitude
      \item Altitude
      \item Regulação litorânea
      \item Correntes Marítimas
      \item Pressão Atmosférica
    \end{itemize}
  \end{block}
\end{frame}


\begin{frame}
  \frametitle{Latitude}
  \begin{center}
    \shadowbox{\includegraphics[scale=0.35]{./figures/latitude.png}}
  \end{center}
\end{frame}


\begin{frame}
  \frametitle{Altitude}
  \begin{center}
    \shadowbox{\includegraphics[scale=0.5]{./figures/altitude.png}}
  \end{center}
\end{frame}


\begin{frame}
  \frametitle{Regulação Litorânea}
  {\small
  São Paulo: máxima 31\textcelsius{}, mínima 20\textcelsius{}\\
  Santos: máxima 33\textcelsius{}, mínima 27\textcelsius{}
  }
  \begin{center}
    \shadowbox{\includegraphics[scale=0.3]{./figures/continentalidade.png}}
  \end{center}
\end{frame}


\begin{frame}
  \frametitle{Correntes Marítimas}
  \begin{center}
    \shadowbox{\includegraphics[scale=0.3]{./figures/gulfstream.png}}
  \end{center}
\end{frame}

\begin{frame}
  \frametitle{Pressão Atmosférica}
  \begin{itemize}[<+-| alert@+>]
    \item Deslocamento sempre da alta para a baixo pressão;
    \item Varia basicamente em função da temperatura:
    \item Alta Temperatura $\rightarrow$ Baixa Pressão;
    \item Baixa Temperatura $\rightarrow$ Alta Pressão;
  \end{itemize}
\end{frame}


\subsection{Massas de Ar}
\begin{frame}
  \frametitle{Massas de Ar}
  \begin{itemize}[<+-| alert@+>]
    \item Características próprias:
    \item Temperatura, Pressão e Umidade;
    \item Identificam o seu lugar de origem.
    \item {\bf Nomes:}
    \item {\bf Ta} $\rightarrow$ Tropical Atlântica
    \item {\bf Tp} $\rightarrow$ Tropical Pacífica
    \item {\bf Ea} $\rightarrow$ Equatorial Atlântica
    \item {\bf Ep} $\rightarrow$ Equatorial Pacífica
    \item {\bf Pa} $\rightarrow$ Polar Atlântica
    \item {\bf Pp} $\rightarrow$ Tropical Pacífica
    \item {\bf Ec} $\rightarrow$ Equatorial Continental
    \item {\bf Tc} $\rightarrow$ Tropical Continental
  \end{itemize}
\end{frame}

\begin{frame}
  \frametitle{Massas de Ar}
  \begin{block}{}
    Um dos fatores mais decisivos na caracterização do clima de uma dada região
    é a atuação das massas de ar, pois ``emprestam'' suas características ao
    tempo e ao clima dos lugares por onde circulam.
  \end{block}
\end{frame}

\begin{frame}
  \frametitle{Massas de Ar}
  \begin{block}{}
    A origem quanto às zonas climáticas determinará a temperatura das massas,
    assim, as que se formarem na zona polar serão frias e as das zonas tropical
    e equatorial, serão quentes.  Da mesma forma, a origem oceânica ou
    continental irá determinar sua umidade que poderá, entretanto, variar com o
    deslocamento da massa por sobre regiões de umidade distinta.
  \end{block}
\end{frame}

\begin{frame}
  \frametitle{Massas de Ar}
  \begin{center}
    \shadowbox{\includegraphics[scale=0.45]{./figures/massa_de_ar.png}}
  \end{center}
\end{frame}

\begin{frame}
  \frametitle{Massas de Ar}
    \begin{columns}
        \begin{column}{0.5\textwidth}
    \begin{center}
        \includegraphics[scale=0.5]{./figures/massa_de_ar-inverno.png}
    \end{center}
        \end{column}
    \begin{column}{0.5\textwidth}
    \begin{center}
        \includegraphics[scale=0.5]{./figures/massa_de_ar-verao.png}
    \end{center}
    \end{column}
    \end{columns}
\end{frame}


\begin{frame}
  \frametitle{1. Massa Equatorial Continental (Ec)}
  {\scriptsize
  \begin{block}{}
    {\bf Ec}: Influencia toda a região do Brasil, principalmente durante o
    verão, quando grande quantidade de umidade são transportadas para a região
    Centro-Sul.
  \end{block}
  }
  \begin{center}
    \shadowbox{\includegraphics[scale=0.45]{./figures/mEc.png}}
  \end{center}
\end{frame}


\begin{frame}
  \frametitle{1. Massa Equatorial Continental (Ec)}
  \begin{block}{}
    É uma massa quente e instável originada na Amazônia Ocidental, que atua
    sobre todas as regiões do país.  Apesar de continental é uma massa úmida, em
    razão da presença de rios caudalosos e da intensa transpiração da massa
    vegetal da Amazônia, região em que provoca chuvas abundantes e quase
    diárias, principalmente no verão e no outono.  No verão, avança para o
    interior do país provocando as ``chuvas de verão''.
  \end{block}
\end{frame}


\begin{frame}
  \frametitle{2. Massa Equatorial Atlântica (Ea)}
  {\scriptsize
  \begin{block}{}
    {\bf Ea}: É quente, úmida, atua nas regiões litorâneas do Norte do Nordeste.
  \end{block}
  }
  \begin{center}
    \shadowbox{\includegraphics[scale=0.45]{./figures/mEa.png}}
  \end{center}
\end{frame}


\begin{frame}
  \frametitle{2. Massa Equatorial Atlântica (Ea)}
  {\small
  \begin{block}{}
    Originária do Atlântico Norte (próximo à Ilha de Açores).  Atua
    principalmente no verão e na primavera, sendo também formadoras dos ventos
    alísios de nordeste.
  \end{block}
  }
\end{frame}

\begin{frame}
  \frametitle{3. Massa Tropical Atlântica (Ta)}
  {\scriptsize
  \begin{block}{}
    {\bf Ta}: Quente e úmida.  Também é formadora dos ventos alísios de sudeste.
  \end{block}
  }
  \begin{center}
    \shadowbox{\includegraphics[scale=0.45]{./figures/mTa.png}}
  \end{center}
\end{frame}


\begin{frame}
  \frametitle{3. Massa Tropical Atlântica (Ta)}
  {\small
  \begin{block}{}
  Origina-se no Oceano Atlântico e atua na faixa litorânea do Nordeste ao Sul do
  país.  Provoca as chuvas frontais de inverno na região Nordeste a partir do
  seu encontro com a Massa Polar Atlântica e as chuvas de relevo nos litorais
  sul e sudeste, a partir do choque com a Serra do Mar.
  \end{block}
  }
\end{frame}


\begin{frame}
  \frametitle{4. Massa Tropical Continental (Tc)}
  {\scriptsize
  \begin{block}{}
    {\bf Tc}: Quente e seca, causa longos períodos quentes e secos no sul da
    região Centro-oeste e no interior das regiões Sul e Sudeste.
  \end{block}
  }
  \begin{center}
    \shadowbox{\includegraphics[scale=0.45]{./figures/mTc.png}}
  \end{center}
\end{frame}


\begin{frame}
  \frametitle{4. Massa Tropical Continental (Tc)}
  {\small
  \begin{block}{}
    Originada na Depressão do Chaco, atua basicamente em sua área de origem.
  \end{block}
  }
\end{frame}


\begin{frame}
  \frametitle{5. Massa Polar Atlântica (Pa)}
  {\scriptsize
  \begin{block}{}
    {\bf Pa}: Fria e úmida, atuando sobretudo no inverno.
  \end{block}
  }
  \begin{center}
    \shadowbox{\includegraphics[scale=0.6]{./figures/mPa.png}}
  \end{center}
\end{frame}


\begin{frame}
  \frametitle{5. Massa Polar Atlântica (Pa)}
  {\small
  \begin{block}{}
    Forma-se no Oceano Atlântico sul (próximo à Patagônia).   No litoral
    nordestino (causa chuvas frontais), nos estados do Sul (causa queda de
    temperatura e geadas) e na Amazônia Ocidental (causa fenômeno da friagem,
    queda brusca na temperatura).
  \end{block}
  }
\end{frame}


\subsection{Resumo}
\begin{frame}
  \frametitle{Resumo}
  \begin{itemize}[<+-| alert@+>]
    \item Massa Equatorial Continental (Ec): Quente e úmida
    \item Massa Equatorial Atlântica (Ea): Quente e úmida
    \item Massa Tropical Atlântica (Ta): Quente e úmida
    \item Massa Tropical Continental (Tc): Quente e seca
    \item Massa Polar Atlântica (Pa): Fria e úmida
  \end{itemize}
\end{frame}

\end{document}
