% Title page.
\title[Aula 02]{Climatologia e Meteorologia}
\subtitle{Aula Introdução}
\author[Filipe Fernandes]{Filipe P. A. Fernandes}
\institute[unimonte]{Centro Universitário Monte Serrat}
\date[Fevereiro 2014]{10 de Fevereiro 2014}

\logo{\includegraphics[scale=0.15]{../common/university_logo.png}}

\begin{document}

% The title page frame.
\begin{frame}[plain]
  \titlepage
\end{frame}

\section*{Outline}
\begin{frame}
\tableofcontents
\end{frame}


\section{Aula 01}
\subsection{Ementa}
\begin{frame}
    \frametitle{Oceanografia Física Descritiva -- Carga horária: 80 h}
    {\bf Ementa} -- Introdução ao estudo do clima.
    {\scriptsize
    \begin{block}{}
    Conceitos e definições: clima e tempo.
    Meteorologia e Climatologia.  Transferência meridional de energia na Terra e
    a formação da circulação geral da atmosfera.  As massas de ar atuantes no
    Brasil.  Interpretação de fenômenos atmosféricos: tipos de massas de ar,
    frentes atuantes, vigor, duração e intensidade das massas de ar na
    retaguarda de frentes polares.  Fundamentos e conceitos de física ambiental
    na atmosfera.  Interpretação de imagens de satélites, acompanhamento das
    condições do tempo.  Os elementos do clima e os fatores geográficos de
    modificação das condições do tempo.  Os elementos do clima e os fatores
    geográficos de modificação das condições iniciais do clima.  Sistemas de
    aquisição de dados meteorológicos: estações clássicas e automáticas.
    Noção de ritmo climático.
    \end{block}
    }
\end{frame}

\begin{frame}
    \frametitle{Bibliografia Básica:}
    {\scriptsize
        \begin{itemize}
        \item VAREJÃO-SILVA, Márcio Adelmo. Meteorologia e climatologia. 2. ed.
              Brasília: INMET, Gráfica e Editora Pax, 2001. 515 p.
        \item LEMES, Marco Antonio Maringolo; MOURA, Antonio Divino. Fundamentos
              de dinâmica aplicados à meteorologia e oceanografia.  Ribeirão
              Preto: Holos , 2002. vix, 296 p. ISBN 8586699330
        \item TUBELIS, Antonio; NASCIMENTO, Fernando José Lino do.  Meteorologia
              descritiva: fundamentos e aplicações brasileiras. São Paulo:
              Nobel, 1988. 373 p. ISBN 8521300077
        \end{itemize}
    }
\end{frame}

\begin{frame}
    \frametitle{Bibliografia Complementar:}
    {\scriptsize
        \begin{itemize}
        \item WAINER, Ilana (Coord.) et al. Atlas climatológico: região
              Atlântico Sul.  São Paulo: Petrobrás , 2012. 82 p.
              ISBN 9788564586215
        \item NATIONAL WEATHER SERVICE. Marine surface wearther observations.
              United States of America: National Oceanic and Atmospheric
              Administration - NOAA, 2002. 154 p. (Obs erving Handbook)
        \item HAZIN, Fabio Hissa Vieira. Meteorologia e sensoriamento remoto,
              oceanografia física, oceanografia química e oceanografia geológica.
              Fortaleza: Martins \& Cordeiro, 2009. 245 p. (Programa Revizee -
              Score Nordeste) ISBN 9788599121108
        \end{itemize}
    }
\end{frame}

\begin{frame}
  \frametitle{Tópicos}
  {\scriptsize
  \begin{block}{}
  \begin{itemize}
  \item[1.] Introdução ao estudo do clima;
  \item[2.] Conceitos e definições: clima e tempo;
  \item[3.] Conceitos e definições: meteorologia e climatologia;
  \item[4.] Transferência meridional de energia na Terra e a formação da
            circulação geral da atmosfera;
  \item[5.] As massas de ar atuantes no Brasil;
  \item[6.] Interpretação de fenômenos atmosféricos: tipos de massas de ar,
            frentes atuantes, vigor, duração e intensidade das massas de ar na
            retaguarda de frentes polares;
  \item[7.] Fundamentos e conceitos de física ambiental na atmosfera;
  \item[8.] Interpretação de imagens de satélites, acompanhamento das condições
            do tempo;
  \item[9.] Os elementos do clima e os fatores geográficos de modificações das
            condições do tempo;
  \item[10.] Os elementos do clima e os fatores geográficos de modificações das
             condições iniciais do clima;
  \item[11.] Sistemas de aquisição de dados meteorológicos: estações clássicas
             e automáticas;
  \item[12.] Noção de ritmo climático.
  \end{itemize}
  \end{block}
  }
\end{frame}

\begin{frame}
\frametitle{Cronograma}
{\scriptsize
  \begin{longtable}[c]{@{}cllr@{}}
  Aula & Data & Conteúdo & Lista/Prova \\
  \midrule\endhead
  01 & 2014-02-03 & NA & \\
  02 & 2014-02-10 & Tópico 1 & \\
  03 & 2014-02-17 & Tópico 2 & \\
  04 & 2014-02-24 & Tópico 3 & \\
  05 & 2014-03-10 & Tópico 4 & \\
  06 & 2014-03-17 & Tópico 5 & \\
  07 & 2014-03-24 & Tópico 6 & \\
  08 & 2014-03-31 & Tópico 7 & \\
  09 & 2014-04-07 & Tópico 1-7 & Revisão P1 \\
  10 & 2014-04-14 & Tópico 1-7 & P1 \\
  11 & 2014-04-28 & Tópico 9 & \\
  12 & 2014-05-05 & Tópico 10 & \\
  13 & 2014-05-12 & Tópico 10 & \\
  14 & 2014-05-19 & Tópico 11 & \\
  15 & 2014-05-26 & Tópico 12 & \\
  16 & 2014-06-02 & Tópico 1-13 & T1 (Seminários) \\
  17 & 2014-06-09 & Tópico 8-12 & P2 \\
  18 & 2014-06-16 & Tópico 1-12 & Revisão PA \\
  19 & 2014-06-30 & Tópico 1-12 & PA \\
  \bottomrule
  \end{longtable}
  }
\end{frame}

\begin{frame}
    \frametitle{Conversa}
    \begin{itemize}
        \item Lista com nomes e e-mails para contato;
        \item Interesses/Áreas/Estágios;
        \item Nível de Inglês;
        \item Nível de Computação;
        \item Nível de Matemática;
    \end{itemize}
\end{frame}

\begin{frame}
    \frametitle{Avaliações}
    \begin{itemize}[<+-| alert@+>]
        \item 35 + 35 + {\bf 30}.
        \item {\bf 30} divididos entre 15 T1 e 15 {\bf TIDIR/Prova Integradora};
        \item O aluno tem direito a uma prova alternativa (com o conteúdo de
              todo o semestre) para a menor nota.
    \end{itemize}
    \pause
    \begin{block}{}
        Não há segunda chamada{\bf *} nem abono{\bf **}
        de falta não amparado por lei!
    \end{block}
    \pause
\end{frame}

\begin{frame}
    \frametitle{Avaliações -- Segunda parte.}
    \begin{itemize}[<+-| alert@+>]
        \item 35 + 35: Serão divididos entre duas provas;
        \item Provas serão realizadas dias 2014-04-14 e 2014-06-09;
        \item O trabalho de 15 pontos (T1) deve ser entregue antes da segunda
              prova (2014-06-02);
        \item Temas para escolher:
    \end{itemize}
\end{frame}

\begin{frame}
    \frametitle{Avaliações -- Segunda parte.}
        \begin{block}{}
        Temas para escolher para o T1 (grupos de 2-3):
        \end{block}
            \begin{enumerate}[<+-| alert@+>]
                \item Zona de convergência do Atlântico Sul;
                \item ENSO;
                \item Sistema de avaliação de qualidade de ar;
                \item Eventos climáticos Bond/Younger Dryas/Heinrich/Pequena Era
                      do Gelo.
            \end{enumerate}
\end{frame}

\end{document}
