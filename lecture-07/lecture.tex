% Title page.
\title[Aula 07]{Climatologia e Meteorologia}
\subtitle{Interpretação de fenômenos atmosféricos}
\author[Filipe Fernandes]{Filipe P. A. Fernandes}
\institute[unimonte]{Centro Universitário Monte Serrat}
\date[Março 2014]{31 de Março 2014}

\logo{\includegraphics[scale=0.15]{../common/university_logo.png}}

\begin{document}

% The title page frame.
\begin{frame}[plain]
  \titlepage
\end{frame}

\section*{Outline}
\begin{frame}
\tableofcontents
\end{frame}

\section{Interpretação de fenômenos atmosféricos}
\subsection{Tipos de Nuvens}
\begin{frame}
\frametitle{Tipos de Nuvens.}
  \begin{itemize}[<+-| alert@+>]
    \item Nuvens Altas: Cirrus (Ci), Cirrostratus (Cs), Cirrocumulus (Cc)
    \item Nuvens Baixas: Stratus (st), Stratocumulus (Sc), Nimbostratus (Ns)
    \item Nuvens Médias: Altostratus (As), Altocumulus (Ac)
    \item Nuvens com desenvolvimento vertical: Culumus (Cu), Cumulonimbus (Cb)
  \end{itemize}
\end{frame}

\begin{frame}
\frametitle{Altas -- Cirrus}
  \begin{center}
    \shadowbox{\includegraphics[scale=0.2]{./figures/cirrus.png}}
  \end{center}
\end{frame}

\begin{frame}
\frametitle{Altas -- Cirrustratus}
  \begin{center}
    \shadowbox{\includegraphics[scale=0.35]{./figures/cirrostratuscomhalo.png}}
  \end{center}
\end{frame}

\begin{frame}
\frametitle{Altas -- Cirrocumulus}
  \begin{center}
    \shadowbox{\includegraphics[scale=0.3]{./figures/cirrocumulus.png}}
  \end{center}
\end{frame}

\begin{frame}
\frametitle{Médias -- Altostratus}
  \begin{center}
    \shadowbox{\includegraphics[scale=0.25]{./figures/altostratus.png}}
  \end{center}
\end{frame}

\begin{frame}
\frametitle{Médias -- Altocumulus}
  \begin{center}
    \shadowbox{\includegraphics[scale=0.25]{./figures/altocumulus.png}}
  \end{center}
\end{frame}


\begin{frame}
\frametitle{Baixas -- Stratus}
  \begin{center}
    \shadowbox{\includegraphics[scale=0.3]{./figures/stratus.png}}
  \end{center}
\end{frame}

\begin{frame}
\frametitle{Baixas -- Stratocumulus}
  \begin{center}
    \shadowbox{\includegraphics[scale=0.3]{./figures/stratocumulus.png}}
  \end{center}
\end{frame}

\begin{frame}
\frametitle{Baixas -- Nimbostratus}
  \begin{center}
    \shadowbox{\includegraphics[scale=0.3]{./figures/nimbostratus.png}}
  \end{center}
\end{frame}

\begin{frame}
\frametitle{Vertical -- Culumus}
  \begin{center}
    \shadowbox{\includegraphics[scale=0.3]{./figures/cumulus.png}}
  \end{center}
\end{frame}

\begin{frame}
\frametitle{Vertical -- Cumulonimbus}
  \begin{center}
    \shadowbox{\includegraphics[scale=0.25]{./figures/cumulonimbus.png}}
  \end{center}
\end{frame}

\begin{frame}
\frametitle{Vertical -- Cumulonimbus}
  \begin{center}
    \shadowbox{\includegraphics[scale=0.25]{./figures/distribuicaonanuvemcb.png}}
  \end{center}
\end{frame}

\begin{frame}
\frametitle{Resumo}
  \begin{center}
    \includegraphics[scale=0.4]{./figures/meteorologyclouds.png}
  \end{center}
\end{frame}

\subsection{Frentes atuantes}


\begin{frame}
\frametitle{Frentes}
\begin{block}{Definição}
  Faixa (ou zona, ou ``superfície'') de transição entre duas massas de ar de
  características diferentes.
\end{block}
\end{frame}


\begin{frame}
\frametitle{Características}
Normalmente é uma região onde:
  \begin{itemize}[<+-| alert@+>]
  \item a pressão tem um valor mínimo relativo;
  \item a temperatura e a umidade variam abruptamente;
  \item os ventos são mais fortes, mudam de direção e são confluentes
  \item ocorre nebulosidade e precipitação.
 \end{itemize}
\end{frame}


\begin{frame}
\frametitle{Características}
  \begin{itemize}[<+-| alert@+>]
  \item A fronteira entre as massas de ar frio e quente sempre se inclina,
        para cima, por sobre a massa de ar frio (que é mais denso);
  \item Quando as massas de ar se deslocam, o ar frio força o ar quente a
        subir, o que provoca a formação de nuvens e precipitação;
  \item Tem larguras de 5 a 50 km, comprimento de 500 a 5000 km, e altura de
        3 a 10 km.
  \end{itemize}
\end{frame}


\subsection{Tipos de Frente}
\begin{frame}
\frametitle{Frente Fria}
\begin{block}{O que é uma Frente Fria (FF)?}
  É uma zona de separação entre duas massas de ar, onde a mais fria avança
  sobre a mais quente.  A frente fria é representada nas cartas sinóticas como
  uma linha azul e triângulos indicando a sua direção.
\end{block}
\end{frame}

\begin{frame}
\frametitle{Frente Fria}
  \begin{center}
    \includegraphics[scale=1]{./figures/simbolofrentefria.png}
  \end{center}
\end{frame}

\begin{frame}
\frametitle{Frente Fria}
  \begin{center}
    \includegraphics[scale=0.35]{./figures/frente_fria.png}
  \end{center}
\end{frame}

\begin{frame}
\frametitle{Frente Fria}
  \begin{center}
    \includegraphics[scale=0.45]{./figures/frente_fria_esquema.png}
  \end{center}
\end{frame}

\begin{frame}
\frametitle{Frente Quente}
\begin{block}{O que é uma Frente Quente (FQ)?}
  É uma zona de transição entre duas massas de ar, uma quente e uma fria, onde
  a mais quente avança sobre a mais fria.  A frente quente é representada nas
  cartas sinóticas como uma linha vermelha e semi-círculos.
\end{block}
\end{frame}

\begin{frame}
\frametitle{Frente Quente}
  \begin{center}
    \includegraphics[scale=1]{./figures/simbolofrentequente.png}
  \end{center}
\end{frame}


\begin{frame}
\frametitle{Frente Quente}
  \begin{center}
    \includegraphics[scale=0.45]{./figures/frente_quente_esquema.png}
  \end{center}
\end{frame}


\begin{frame}
\frametitle{Frente Estacionária}
\begin{block}{O que é uma Frente Estacionária?}
  É a frente entre uma massa de ar quente e uma fria, que esta se movimento
  muito lentamente, ou não possui movimentação nenhuma.  A frente estacionária
  é representada nas cartas sinóticas com os símbolos da FF e FQ alternados em
  direções alternadas.
\end{block}
\end{frame}


\begin{frame}
\frametitle{Frente Estacionária}
  \begin{center}
    \includegraphics[scale=1]{./figures/simbolofrenteestacionaria.png}
  \end{center}
\end{frame}


\begin{frame}
\frametitle{Frente Oclusa}
\begin{block}{O que é uma Frente Oclusa?}
  Ela é composta por duas frentes, onde a frente fria avança sobre a frente
  quente ou uma frente quase-estacionária.  Existem dois tipos, a oclusão fria
  que ocorre quando o ar frio esta atrás da frente fria e a oclusão quente, que
  ocorre quando o ar frio esta atrás da frente quente.  A frente oclusa é
  representada nas cartas sinóticas com os símbolos da quente e fria alternadas,
  mas na mesma direção e com uma coloração roxa.
\end{block}
\end{frame}


\begin{frame}
\frametitle{Frente Oclusa}
  \begin{center}
    \includegraphics[scale=1]{./figures/simbolofrenteoclusa.png}
  \end{center}
\end{frame}

\begin{frame}
\frametitle{Frente Oclusa/Estacionária}
  \begin{center}
    \includegraphics[scale=0.5]{./figures/frente_oclusa_estacionaria_esquema.png}
  \end{center}
\end{frame}

\begin{frame}
\frametitle{Ordem de Grandeza das Frentes}
  \begin{center}
    \includegraphics[scale=0.5]{./figures/ordem_grande.png}
  \end{center}
\end{frame}


\begin{frame}
\frametitle{Outras linhas de carta sinótica}
  \begin{itemize}[<+-| alert@+>]
    \item Cavado representa uma região de baixa pressão atmosférica.  Já a
          crista será o oposto.
    \item Instabilidade: linha de tempestades ativas.  Elas incluem áreas de
          grande precipitação, resultantes das áreas de tempestade.
    \item Linha Seca: Separação entre ar úmido e seco.
  \end{itemize}
\end{frame}


\subsection{Fenômenos Atmosféricos}
\begin{frame}
\frametitle{Fenômenos}
  \begin{itemize}[<+-| alert@+>]
  \item Tempestade: Designa as nuvens que possuem raios e/ou trovões. Estão
        associadas a rajadas de vento, chuva e/ou granizo.
  \item Ventania: Intenso fluxo de ar.
  \item Rajada: Fluxo instantâneo de ar.
  \item Granizo: ``Pedra'' de gelo formado a partir da colisão de água super
        resfriada e cristais de gelo.
  \item A nuvem associada a todos estes fenômenos é a Cumulonimbus: Latim
        ``acumular'' + Nimbus: Latim ``nuvem de chuva'').
\end{itemize}
\end{frame}


\begin{frame}
\frametitle{Fenômenos}
  \begin{itemize}[<+-| alert@+>]
  \item Raio/Relâmpagos: São sinônimos e são descargas elétricas na
        atmosféricas.  Normalmente usa-se o termo raio para designar as
        descargas que atingem a superfície e o termo relâmpago para designar as
        descargas que ocorrem entre nuvens e não atingem a superfície.
  \item Trovão: O raio aquece a atmosfera ao seu redor, causando um brusco
        aquecimento e uma rápida expansão do ar, consequentemente o ruído.
\end{itemize}
\end{frame}


\begin{frame}
\frametitle{Fenômenos}
\begin{itemize}[<+-| alert@+>]
  \item Tornado: A partir da base da nuvem, forma-se um funil em direção ao
        solo, denominado de tornado.
  \item Tromba d'Água: Tornado acima de uma grande superfície de água.
\end{itemize}
\end{frame}

\begin{frame}
\frametitle{Fenômenos}
\begin{itemize}[<+-| alert@+>]
  \item Furacão: É a maior tempestade do planeta.  Pode atingir diâmetros de até
        mais de 1000 km. Só se formam onde a temperatura da água esta acima de
        27\textcelsius{}, pois necessitam de muito vapor para se manter.
  \item Nota: Ciclone Tropical, Furacão e Tufão são palavras que correspondem ao
        mesmo tipo e fenômeno. Dependendo da região de ocorrência, o fenômeno
        pode apresentar nomes diferentes.
\end{itemize}
\end{frame}


\subsection{Furacões, Tufões e Ciclones Tropicais}
\begin{frame}
\frametitle{O que são?}
\begin{itemize}[<+-| alert@+>]
  \item Centro de baixa pressão não-frontal;
  \item Escala sinótica;
  \item Convecção organizada;
  \item Intensa circulação ciclônica na superfície;
  \item Associados às Ondas de Leste.
\end{itemize}
\end{frame}


\begin{frame}
\frametitle{Classificação}
\begin{table}[h]
  \begin{tabular}{l|l}
  \hline\hline
  {\bf Vento máximo em superfície}       & {\bf Fenômeno}        \\
  Menor que 17 ms$^{-1}$                 & Depressões Tropicais  \\
  17 ms$^{-1}$ a 32 ms$^{-1}$            & Tempestades Tropicais \\
  33 ms$^{-1}$ ou maior                  & Furacões, Tufões etc. \\
  \hline
  \end{tabular}
\end{table}
\end{frame}

\begin{frame}
\frametitle{Escala Saffir-Simpson}
{\small
\begin{table}[h]
  \begin{tabular}{|l|l|l|l|}
  \hline
  {\bf Categoria} & {\bf Vento (m s$^{-s}$)} & \bf {Pressão (mb)} & {\bf Surge (m)}\\  \hline \hline
  1 &  33--42 & $>$980   & 1,0--1,7 \\ \hline
  2 &  43--49 & 979--965 & 1,8--2,6 \\ \hline
  3 &  50--58 & 964--945 & 2,7--3,8 \\ \hline
  4 &  59--69 & 944--920 & 3,9--5,6 \\ \hline
  5 &  $>$70  & $<$920   & $>$5,7   \\ \hline
  \end{tabular}
\end{table}
}
\end{frame}

\begin{frame}
\frametitle{Ondas de Leste}
\begin{itemize}[<+-| alert@+>]
  \item Distúrbios que se propagam para oeste imersos no escoamento dos
        Alíseos, da superfície do mar até 5km, com máximo em 3km;
  \item Ondas geradas por instabilidades no African Easterly Jet (norte da
        África);
  \item Ocorrem de Abril/Maio até Outubro/Novembro;
  \item Parte convectivamente ativa ao longo de um trem de ondas;
  \item Duram de 3 a 4 dias, com comprimentos de onda entre 2 e 2,5 mil km.
\end{itemize}
\end{frame}


\begin{frame}
\frametitle{Ondas de Leste}
  \begin{center}
    \includegraphics[scale=0.45]{./figures/ondas_leste.png}
  \end{center}
\end{frame}

\begin{frame}
\frametitle{Ciclones: Tropicais vs Extratropicais}
{\scriptsize
\begin{table}[h]
  \begin{tabular}{l|l}
  \hline\hline
  {\bf Ciclone Tropical} & {\bf Ciclone Extratropical} \\
  Furacão, Tufão, Tempestade Tropical & Ciclone frontal, Baixa Extratropical\\
  Energia da evaporação & Energia do gradiente latitudinal de temperatura \\
  Núcleo quente em baixos níveis & Núcleo frio em baixos níveis \\
  Ventos intensos próximos à superfície & Ventos intensos próximos à tropopausa \\
  \hline
  \end{tabular}
\end{table}
}
\end{frame}

\begin{frame}
\frametitle{Ciclones vs Tornado}
{\scriptsize
\begin{table}[h]
  \begin{tabular}{l|l|l}
  \hline\hline
                       & {\bf Ciclone Tropical} & {\bf Tornado} \\
  \hline
  {\bf Diâmetro}       & Centenas de km         & Centenas de metros \\
  {\bf Ciclo de Vida}  & Dias                   & Minutos            \\
  {\bf Proporção}      & Dezenas de tempestades & Formado por uma tempestade \\
  {\bf Cisalhamento Vertical} & Baixo           & Alto  \\
  {\bf Local de Formação}     & Oceanos         & Continente (principalmente) \\
  \hline
  \end{tabular}
\end{table}
}
\end{frame}

\begin{frame}
\frametitle{Ciclones Tropicas -- Formação}
  \begin{itemize}[<+-| alert@+>]
    \item Águas quentes (~26,5\textcelsius{}) a ~50 m de profundidade, no mínimo;
    \item Atmosfera que se resfrie rapidamente com a altura (instável à convecção
          úmida);
    \item Umidade em níveis médios;
    \item Distância de ~500 km do Equador (força de Coriolis não desprezível);
    \item Distúrbio pré-existente em superfície (convergência e vorticidade
          suficientes);
    \item Baixo cisalhamento vertical do vento.
  \end{itemize}
\end{frame}


\begin{frame}
\frametitle{Ciclones Tropicas -- Fatores de larga escala}
  \begin{block}{ENSO (El Ni\~no Southern Oscillation)}
    Fase quente (El Niño): aumenta o  Cisalhamento vertical, inibindo a
    formação de ciclones tropicais.\\

    Fase fria (La Niña): indicativo de aumento destes ciclones.
  \end{block}
\end{frame}

\begin{frame}
\frametitle{Atlântico Sul...}
  \begin{block}{}
    Por quê não é normal a ocorrência de Ciclones Tropicais no Atlântico Sul?
  \end{block}
% - Forte cisalhamento do vento entre
% a superfície e a alta troposfera;
% - ITCZ tipicamente inexiste sobre
% o Atlântico Sul;
% - TSM não é tão elevada.
\end{frame}


\subsection{Ciclones Extra-Tropicais}
\begin{frame}
\frametitle{Ciclone Extra-Tropicais (ou de latitudes médias)}
  \begin{block}{Definição}
    Área de baixa pressão,  na forma de um núcleo fechado, onde os ventos
    giram no sentido.
  \end{block}
\end{frame}


\begin{frame}
\frametitle{Ciclones Extra-Tropicais}
  \begin{center}
    \includegraphics[scale=0.4]{./figures/extra_tropical.png}
  \end{center}
\end{frame}


\begin{frame}
\frametitle{Ciclones Extra-Tropicais}
  \begin{center}
    \includegraphics[scale=0.65]{./figures/convergencia_frentes.png}
  \end{center}
\end{frame}


% \begin{frame}
% \frametitle{Ciclones Extra-Tropicais (Exemplos)}
%   \begin{center}
%     \includegraphics[scale=0.45]{./figures/extratropicais_01.png}
%   \end{center}
% \end{frame}
%
%
% \begin{frame}
% \frametitle{Ciclones Extra-Tropicais (Exemplos)}
%   \begin{center}
%     \includegraphics[scale=0.45]{./figures/extratropicais_02.png}
%   \end{center}
% \end{frame}


\begin{frame}
\frametitle{Fenômenos Regionais}
  \begin{center}
    \includegraphics[scale=0.45]{./figures/ilha_de_calor.png}
  \end{center}
\end{frame}

\begin{frame}
\frametitle{Fenômenos Regionais}
  \begin{center}
    \includegraphics[scale=0.4]{./figures/inversao_termica.png}
  \end{center}
\end{frame}

\begin{frame}
\frametitle{Ciclone Extra-Tropical Catarina}
  \begin{block}{Images}
    Discutir..
  \end{block}
\end{frame}

\end{document}
